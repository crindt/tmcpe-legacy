% \section*{Overview}
% \label{execsum-overview}

This document reports on the results of contract \#65A0252, \emph{A
  Measurement-based System for TMC Performance Evaluation}.  This
project developed a TMC performance evaluation system that allows TMC
managers to evaluate the performance of various bundles of TMC
technologies and operational policies by mapping their effects onto
events in the system that can be measured using existing surveillance
systems and daily activity logs.  The work included tasks to develop a
method for quantifying the relative benefits of TMC operations as well
as tasks to provide a user friendly interface for performing benefits
analysis of past incidents using available data sources.  The system
was deployed atop the CTMLabs web architecture and is available to
approved users as part of the secure, authenticated, CTMLabs website.

\section*{Completed Work}
\label{execsum-sum-of-work}

In Q2-2008, the team began a discovery process to determine resources and data
available for implementing the performance evaluation system. We identified the
major data available (freeway loop data, TMC activity logs, CHP CAD feed) as
core sources and obtained representative samples for analysis.  We also started
a review of the performance evaluation methodology with the input of Caltrans
TMC personnel.

In Q3-2008, we settled on the delay calculation methodology to be used as the
primary analysis tool for the performance evaluation project.  As part of this
process, we developed a prototype software implementation using \texttt{perl}
scripting that takes inputs from local copies of PeMS freeway data and the TMC
activity log databases, generates a mathematical program to determine the
impacted freeway region for given incidents, and solves that program using the
\texttt{GAMS} solver.  This software was used to successfully analyze a set of
incidents identified by D12.

In Q4-2008, the team worked on finalizing the complete performance evaluation
system prototype. This work included creating the ability to accept incident
data from the CHP CAD XML feed, which is archived for cross referencing with the
Caltrans activity log.  Further work was carried out on encapsulating the core
delay calculation algorithm so that it can be integrated into a user-driven
performance evaluation application.  Discussions during the quarter with the TMC
team revealed that modifications can be made to the activity log in order to
better support the performance evaluation system.  The team engaged in a
fact-finding exercise to better understand the current TMC architecture as well
as the likely evolution of that system going forward. Based upon these
assessments, the team developed a set of recommendations for upgrading the
activity log.

Work during Q1-2009 was disrupted by a suspension of this contract due to
California's budget crisis.  Work carried out prior to the suspension of the
contract focused on linking all information sources for the core performance
evaluation engine developed in earlier work.  A prototype web-based interface
for viewing performance evaluation results was developed, but this interface has
not yet been connected to the analytical back end due to the suspension of work
on the contract.  Additional work included developing interfaces for ITS's ATMS
database, which will replace the static copy of sensor data that was used in
initial development. Switching to the ATMS database will eventually speed
deployment in the TMC.

Work during Q2-2009 continued on two fronts: developing and deploying a
web-based interface for viewing performance evaluation results and the back-end
system that will perform the performance evaluation. The prototype visually
displays incident conditions and estimated impacted regions using a time-space
block diagram of impacted facilities, similar to the figures generated in the
original research. Work also continued on developing an efficient database
containing rolling 1-year of historical sensor measurements from the ITS ATMS
database. Work on linking the back-end data sources and the interface prototype
was carried out off-line, but used data formats (JSON) that will be identical to
those eventually produced by the back-end processing. Efforts to directly link
the live data sources (ATMS, TMC activity log, and the CHP CAD XML feed) to the
front end were put on hold until the revisions to the log discussed during
November 2008 are completed by subcontractors during Q2-2009.

Work during Q3-2009 focused on three areas.  On the modeling side, signficant
progress was made in developing a more rigorous approach to converting sensor
data into evidence of incident impacts.  This work uses more advanced
statistical methods to identify behavior at each sensor that differs from mean
patterns.  It is anticipated that this work will replace the existing
threshold-based approach to generating evidence.  Work also continued on
readying the database to support the performance evaluation model and interface.
These efforts are being synchronized with work on the related Safety project
that relies on similar datasets.  Additionally, progress was made on developing
a consistent analytical representation of the D12 network to support the
modeling of network effects.  Toward this end, the freely available
openstreetmap dataset was imported into the Testbed database for use in modeling
and visualization.  This will give the Testbed a map dataset that is
unencumbered by license restrictions and which can be extended to meet Testbed
needs in the future.  Finally, work on the front-end interface continued,
including the preliminary development of a map-based front end for viewing
incident histories.

In parallel with the above efforts, the team continued to collaborate on the
development of the activity log upgrade which will better support performance
evaluation as it goes live.  The team also engaged in a fact finding interview
with TMC staff to document TMC processes used during incident management.


\section*{Ongoing and Planned Work}
\label{execsum-ongoing-work}

Going forward, work is slated to continue in several areas.  We are exploring
some better methods to use in the incident impact model
%
\iffull
(see section~\ref{sec:incident-impact-model})
\fi
%
for identifying when traffic conditions are statistically significant from the
norm.  Notably, we are working with members of the Department of Computer
Science to apply Bayesian techniques to the analysis of traffic state data.  We
anticipate that these approaches will prove more robust than the course
techniques currently being used.

In parallel to the above effort, we will begin developing an implementation of
the
%%
\iffull%
TMC impact model (described in section~\ref{sec:mod-tmc-impacts}).%
\else%
TMC impact model.%
\fi
%
In this task, we will use a derived theoretical relationship between response
time and delay magnitude and compare these predictions to the estimated total
delays from the incident impact model for a subset of incidents managed by the
D12 TMC.  The goal is to demonstrate that the theoretical relationship generally
holds and can therefore be used as a basis for predicting TMC benefits.

Finally, we will also bring the TMC interface to the performance evaluation
system on-line.  Initially, this will focus on the display of
%
\iffull%
previously analyzed incidents (a prototype interface is shown in section~\ref{sec:interfaces}).%
\else%
previously analyzed incidents.%
\fi%
%
Fundamental to this effort is integration of the incident impact and TMC impact
models into the Testbed's Enterprise Service Bus architecture that is currently
under development.  Furthermore, the system must be connected to the two main
sources of data that support the analysis: the ATMS database (or UCI's copy of
it) and the activity log database (or a shadowed version of it).  Once these
components are integrated, the web-based interface can be connected to live data
and released for initial testing by potential users.


% Going forward, we will make the prototype web interface available on a
% limited basis for analysis of incidents from the latter half of 2007
% (for which we have data from the D12 activity log). The purpose of
% this delivery is to get feedback from Caltrans regarding the interface
% and the type of analysis data made available.

% At the same time, once the revisions to the activity log are
% completed, we will finalize plans to integrate the system with TMC
% activity log data sources and begin work on a live version of the
% web-based prototype for recent incidents that incorporates feedback
% offered regarding the interface.
