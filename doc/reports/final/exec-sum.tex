% \section*{Overview}
% \label{execsum-overview}

This research project developed a web-based \ac{TMCPE} application that
addresses the problem of identifying the value of the \ac{TMC} in managing
disruptions to the transportation system.  To achieve this goal, Caltrans needs
a method that:
\begin{itemize}
\item uses available datasources,
\item makes direct inferences from available data without the use of simulation,
  and
\item can be converted directly into dollar values.
\end{itemize}

Though a range of techniques are available for valuing the \ac{TMC}, the
research team focused quantifying the delay savings that can be attributed
directly to \ac{TMC} actions.  Using event data from \ac{TMC} activity logs and
traffic state data from the \ac{PeMS} database, the technique developed first
identifies the time-space impact of events in the activity database using a
mathematical-programming formulation to match evidence of disruption to computed
time-space bounds.  Given this boundary, the actual delay associated with the
impacted region is calculated.  To compute the savings attributable to the
\ac{TMC}, the activity logs are used to identify when the direct disruption by
the event is removed (e.g., when an accident is cleared) and models the
increased delay that would occur if this clearance was delayed.

Given these calculations, the system allows \ac{TMC} managers to evaluate the
performance of various bundles of \ac{TMC} technologies and operational policies
by mapping their effects onto events in the system that can be measured using
existing surveillance systems and daily activity logs. The resulting tool
provides managers with the long needed capabilities to:
\begin{itemize}
\item justify valuable technology, personnel allocations, and
  maintenance costs,
\item identify technologies that aren't meeting their initial promise,
  and
\item identify gaps in current operational strategies that might be
  filled with new technology deployments.
\end{itemize}

The system was deployed atop the \ac{CTMLabs} web architecture and is available
to approved users as part of the secure, authenticated, \ac{CTMLabs} website.
The resulting \ac{TMCPE} website allows users to query the analyzed incident
database to obtain general statistics about \ac{TMC} performance as well as view
the detailed analysis of each incident in the system.

